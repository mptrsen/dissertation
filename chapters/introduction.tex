%!TEX root = ../dissertation.tex
\chapter{Introduction}
\label{introduction}

We are currently experiencing a mass extinction event that parallels other episodes in earth's history with high rates of biodiversity decline \citep{Pimm1995, Dirzo2003, Schipper2008, Barnosky2011, Dirzo2014}. Other than the five extinction events preceding it \citep{Kolbert2014}, it is anthropogenic in origin \citep{Leakey1996, Ceballos2015} and is associated with global warming \citep{Cook2016, Wuebbles2017}, large-scale deforestation \citep{Wright2005}, destruction of marine and freshwater habitats \citep{Burkhead2012}, and introduction of invasive species \citep{Mooney2001}, all hallmarks of human influence. Put shortly, the rate at which species go extinct is alarming \citep{Newbold2016, Ceballos2017, Hallmann2017}, and our children will likely experience a world with less than half the biodiversity we know today. While this issue has raised the attention of country leaders and conservation policies are being put in place worldwide \citep{Puntaru2017}, this might not be enough to reverse the trend without sustaining irreparable damage to the ecosystems of the planet. To make matters worse, there are signs that the issue, despite its urgency, is fading from public awareness \citep{Kusmanoff2017}. 

Conservation efforts require intimate knowledge of the systems they want to preserve: Of course, we cannot save what we do not know. The road towards understanding the biology and the interaction of species is, however, traveled on multiple levels. It is not enough to observe the behaviour or the feeding preferences of an animal to understand the impact of it being removed from its habitat. It is also not enough to describe functional morphology to gain insight on ecological implications. Neither is it sufficient to study the genes and draw conclusions based on their composition and structure. Profound understanding of any system can only be gained by studying it from multiple angles and with interdisciplinary approaches. One such approach is to sequence and analyse the genome of a species: the source code of life that defines, by a manifold of means, its appearance, features, behaviour and interactions with the environment.

%!TEX root = ../dissertation.tex
\chapter{General Conclusion}
\label{conclusion}

The research that comprises the present thesis has added to the body of
current knowledge on both methodological and empirical levels. Through
the development of software for a broad range of disciplines, it
provides a future-proof tool that enables researchers to reliably infer
orthology among genes and transcripts. By characterizing the insect TE
repertoire and its influence on the dynamics of genome size evolution,
it sheds light on the impact that TE activity can have on genome
architecture.


\section{Roadmap}

Orthograph (chapter \ref{cha:orthograph}) provides versatile tool ---
not only for phylogenetics but also for other disciplines --- has been
used in many studies --- mostly phylogenetics, but also evolutionary
biology (Gómez-Zurita) --- not limited to transcriptomes but also useful
with HE data --- etc --- phylogenetics important because provides the
backbone phylogeny that is required for all evolutionary inferences ---
in chapter \ref{cha:mobilome} the insect mobilome is characterized ---
etc


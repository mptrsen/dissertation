%!TEX root = ../dissertation.tex
\begin{savequote}[75mm]
This chapter is intended for publication in BMC Evolutionary Biology.

Authors: Malte Petersen, David Armisén, Richard A. Gibbs, Lars Hering,
Abderrahman Khila, Georg Mayer, Stephen Richards, Oliver Niehuis, and
Bernhard Misof
%\qauthor{Firstname lastname}
\end{savequote}

\chapter{Diversity and evolution of the transposable element
repertoire in insects}
\label{cha:mobilome}

\section{Introduction}\label{introduction}

Repetitive elements, including transposable elements (TEs), are a major
sequence component of eukaryote genomes. In vertebrate genomes, for
example, the TE content varies from 6 \% in the pufferfish
\emph{Tetraodon nigroviridis} to more than 55 \% in the zebrafish
\emph{Danio rerio} \citep{Chalopin2015}. More than 45 \% of the human
genome \citep{deKoning2011} consist of TEs. In plants, TEs are even more
prevalent: up to 90 \% of the maize (\emph{Zea mays}) genome is covered
by TEs \citep{SanMiguel1996}. In insects, the genomic portion of TEs ranges
from as little as 1 \% in the antarctic midge \citep{Kelley2014} to as
large as 65 \% in the migratory locust \citep{Wang2014}.

TEs are known as ``jumping genes'' and traditionally viewed as selfish
parasitic nucleotide sequence elements propagating in genomes with
mainly deleterious or at least neutral effects on host fitness
\citep{Mackay_1989}. Due to their propagation in the genome, TEs are
thought to have a considerable influence on the evolution of the host's
genome architecture. By transposing into, for example, host genes or
regulatory sequences, TEs can disrupt coding sequences or gene
regulation, and/or provide hot spots for ectopic (non-homologous)
recombination that may induce chromosomal rearrangements in the host
genome such as deletions, duplications, inversions, and translocations
\citep{Burns2012}. For example, the shrinkage of the Y chromosome in the
fruit fly \emph{Drosophila melanogaster}, which consists mostly of TEs,
is thought to be caused by such intrachromosomal rearrangements induced
by ectopic recombination \citep{Adams2000}. As such potent agents for
mutation, TEs are also responsible for cancer and genetic diseases in
humans and other organisms
\citep{Vorechovsky2009,Chenais2015,Hancks2016}.

Despite the potential deleterious effects of their activity on gene
regulation, there is growing evidence that TEs can also be drivers of
genomic innovation generating selective advantages to the host
\citep{Casola2007,Gonzalez2008}. For example, it is well documented that
the frequent cleavage and rearrangement of DNA strands induced by TE
insertions provides a source of sequence variation to the host genome,
or that by a process called molecular domestication of TEs, host genomes
derive new functional genes and regulatory networks
\citep{Feschotte2008,Boehne2008,Santos_2014}.  Furthermore, many exons
have been \emph{de novo}-recruited from TE insertions in coding
sequences of the human genome \citep{Zhang2006}.  In insects, TE
insertions have played a pivotal role in the acquisition of insecticide
resistance \citep{Chen2007,Itokawa2010,Gahan2001}, as well as in the
rewiring of a regulatory network that provides dosage compensation
\citep{Ellison2013}, or the evolution of climate adaptation
\citep{Gonzalez2010,Kim2014}.



TEs are classified depending on their mode of transposition. Class I
TEs, also known as retrotransposons, transpose via an RNA-mediated
mechanism that can be circumscribed as ``copy-and-paste''. They are
further subdivided into long terminal repeat (LTR) retrotransposons and
non-LTR retrotransposons. Non-LTR retrotransposons include long and
short interspersed nuclear elements (LINEs and SINEs)
\citep{Malik1999,Eickbush2008}. Whereas LTR retrotransposons and LINEs
encode a reverse transcriptase, the non-autonomous SINEs rely on the
transcriptional machinery of autonomous elements, such as LINEs, for
mobility. Frequently found LTR retrotransposon families in eukaryote
genomes include Ty3/Gypsy, which was originally described in
\emph{Arabidopsis thaliana} \citep{Marin2000}, Ty1/Copia
\citep{Flavell1992}, as well as BEL/Pao \citep{de_la_Chaux2011}.

In Class II TEs, also termed DNA transposons, the transposition is
DNA-based and does not require an RNA intermediate. Autonomous DNA
transposons encode a transposase enzyme and move via a ``cut-and-paste''
mechanism. During replication, terminal inverted repeat (TIR)
transposons and Crypton-type elements cleave both DNA strands
\citep{Wicker2007}. Helitrons, also known as rolling-circle (RC)
transposons due to their characteristic mode of transposition
\citep{Kapitonov2001}, and Maverick elements \citep{Kapitonov2006}
cleave a single DNA strand in the process of replication. Class II also
encompasses non-autonomous DNA transposons such as miniature inverted
TEs (MITEs) \citep{Shirasawa2012}, which exploit and rely on the
transposase mechanisms of autonomous DNA transposons to replicate.

Previous reports on insect genomes describe the composition of TE
families in insect genomes as a mixture of metazoan and insect specific
TEs \citep{Feschotte2007}. Dedicated comparative analyses of TE
composition, however, have only been conducted on species of mosquitoes
\citep{Neafsey2014} and of drosophilid flies \citep{Sessegolo2016}.
Overall, surprisingly little effort has been put into characterizing TE
sequence families and TE compositions in insect genomes. While this lack
of knowledge is due to the low availability of sequenced insect genomes
in the past, efforts such as the i5k initiative \citep{Robinson2011}
have helped to increase the number of genome sequences from previously
unsampled insect taxa. With this denser sampling of insect genomic
diversity available, it now seems possible to comprehensively
investigate the TE diversity among major insect lineages.

Here, we present the first exhaustive analysis of the distribution of TE
classes in a sample representing half of the currently classified insect
orders and using standardized comparative methods implemented in
recently developed software packages.  Our results show similarities in
TE family diversity and abundance among the investigated insect genomes,
but also profound differences in TE activity even among closely related
species.

\section{Materials and methods}\label{materials-and-methods}

\subsection{Genomic data sets}\label{genomic-data-sets}

We downloaded genome assemblies of 42 arthropod species from NCBI
GenBank at \url{ftp.ncbi.nlm.nih.gov/genomes} (last accessed 2014-11-26;
supplementary table S2) as well as the genome assemblies of 31
additional species from the i5k FTP server at
\href{ftp://ftp.hgsc.bcm.edu:/I5K-pilot/}{ftp.hgsc.bcm.edu/I5K-pilot}
(last accessed 2016-07-08; supplementary table S2). Our taxon sampling
includes 21 dipterans, four lepidopterans, one trichopteran, five
coleopterans, one strepsipteran, 14 hymenopterans, one psocodean, six
hemipterans, one thysanopteran, one blattodean, one isopteran, one
orthopteran, one ephemeropteran, one odonate, one archaeognathan, and
one dipluran. As outgroups we included three crustaceans, one myriapod,
six chelicerates, and one onychophoran.

\subsection{Construction of species-specific repeat libraries and TE
annotation in the
genomes}\label{construction-of-species-specific-repeat-libraries-and-te-annotation-in-the-genomes}

We compiled species-specific TE libraries using automated annotation
methods. RepeatModeler Open-1.0.8 \citep{Smit2015} was employed to
cluster repetitive \emph{k}-mers in the assembled genomes and infer
consensus sequences. These consensus sequences were classified using a
reference-based similarity search in RepBase Update 20140131
\citep{Jurka2005}. The entries in the resulting repeat libraries were
then searched for using nucleotide BLAST in the NCBI nr database
(downloaded 2016-03-17 from
\href{ftp://ftp.ncbi.nlm.nih.gov/blast/db}{ftp.ncbi.nlm.nih.gov/blast/db})
to verify that the included consensus sequences are indeed TEs and not
annotation artifacts. Repeat sequences that were annotated as
``unknown'' and that resulted in a BLAST hit for known TE proteins such
as reverse transcriptase, transposase, integrase, or known TE domains
such as gag/pol/env, were kept and considered unknown TE nucleotide
sequences; but all other ``unknown'' sequences were not considered TE
sequences and therefore removed. The filter patterns are listed in
supplementary table S3. The filtered repeat library was combined with
the Metazoa-specific section of RepBase version 20140131 and
subsequently used with RepeatMasker 4.0.5 \citep{Smit2015} to annotate
TEs in the genome assemblies.

\subsection{Validation of Alu
presence}\label{validation-of-alu-presence}

To exemplarily validate our annotation, we selected the SINE Alu, which
was previously only identified in primates \citep{Kriegs2007}. We
retrieved a Hidden Markov model (HMM) profile for the AluJo subfamily
from the repeat database Dfam \citep{Hubley2016} and used the HMM to
search for Alu copies in the genome assemblies. We extracted the hit
nucleotide subsequences from the assemblies and inferred a multiple
nucleotide sequence alignment with the canonical Alu nucleotide sequence
from Repbase \citep{Jurka2005}.

\subsection{Genomic TE coverage and correlation with genome
size}\label{genomic-te-coverage-and-correlation-with-genome-size}

We used the tool ``one code to find them all'' \citep{Bailly-Bechet2014}
on the RepeatMasker output tables to calculate the genomic proportion of
annotated TEs. ``One code to find them all'' is able to merge entries
belonging to fragmented TE copies to produce a more accurate estimate of
the genomic TE content and especially the copy numbers. To test for a
relationship between genome assembly size and TE content, we applied a
linear regression model and tested for correlation using the Spearman
rank sum method. To see whether the genomes of holometabolous insects
are different than the genomes of hemimetabolous insects in TE content,
we tested for an effect of the taxa using their mode of metamorphosis as
a three-class factor: Holometabola (all holometabolous insect species),
non-Eumetabola (all hemimetabolous insect species and non-insect
outgroup species, with the exception of Hemiptera and Psocodea;
\citet{Beutel_2013}), and Acercaria (Hemiptera and Psocodea). We also
tested for a potential phylogenetic effect on the correlation between
genome size and TE content with the phylogenetic independent contrasts
(PIC) method proposed by \citet{Felsenstein1985} using the ape package
\citep{Paradis2004} within R \citep{RCoreTeam2017}





\subsection{Kimura distance-based TE age
distribution}\label{kimura-distance-based-te-age-distribution}

We used intra-family TE nucleotide sequence divergence as a proxy for
intra-family TE age distributions. Sequence divergence was calculated as
intra-family Kimura distances (rates of transitions and transversions)
using the specialized helper scripts from the RepeatMasker 4.0.5
package. We did not use the CpG pair corrected Kimura distances since TE
DNA methylation is clearly absent in holometabolous insects and
insufficiently described in hemimetabolous insects \citep{Glastad2014}.
All TE age distribution landscapes were inferred from the data obtained
from annotating the genomes with \emph{de novo}-generated
species-specific repeat libraries.

\section{Results}\label{results}

\subsection{Diversity of TE content in arthropod
genomes}\label{diversity-of-te-content-in-arthropod-genomes}

TE content varies greatly among the analyzed species (Fig. 1,
supplemental table S1) and differs even between species belonging the
same order. In the insect order Diptera, for example, the TE content
varies from around 55 \% in the yellow fever mosquito \emph{Aedes
aegypti} to less than 1 \% in \emph{B. antarctica}. Even among closely
related \emph{Drosophila} species, the TE content ranges from 40 \% (in
\emph{D. ananassae}) to 10 \% (in \emph{D. miranda} and \emph{D.
simulans}). The highest TE content (60 \%) was found in the large genome
(6.5 Gbp) of the migratory locust \emph{Locusta migratoria}
(Orthoptera), while the smallest known insect genome, that of the
antarctic midge \emph{Belgica antarctica} (Diptera, 99 Mbp), was found
to contain less than 1 \% TEs. The TE content of the majority of the
genomes was spread around a median of 24.4 \% with a standard deviation
of 12.5 \%.

\begin{figure}[h!]
\begin{center}
\includegraphics[width=\textwidth]{te-coverage-in-insect-genomes-both}
\caption[TE coverage in arthropod genomes]{{Genome assembly size, total
amount and relative proportion of DNA transposons, LTR, LINE and SINE
retrotransposons in arthropod genomes and a representative of
Onychophora as an outgroup. Also shown is the genomic proportion of
unclassified/uncharacterized repetitive elements.  Pal., Palaeoptera%
}}
\end{center}
\end{figure}

\subsection{Contribution of TEs to insect genome
size}\label{contribution-of-tes-to-insect-genome-size}

We assessed the TE content, that is, the ratio of TE versus non-TE
nucleotides in the genome assembly, in 62 insect species as well as 10
non-insect arthropods and a representative of Onychophora (velvet worms)
as an outgroup. We tested whether there was a relationship between TE
content and genome assembly size, and found a positive correlation (Fig.
2 and supplemental table S1). This correlation is statistically
significant (Spearman's rank sum test, \(p \lll 0.005\)). Genome size is
significantly smaller in holometabolous insects than in
non-holometabolous insects and Acercaria (one-way ANOVA,
\(p = 0.0001\)). Using the \texttt{ape} package v. 4.1
\citep{Paradis2004} for R \citep{RCoreTeam2017}, we tested for correlation
between TE content and genome size using phylogenetically independent
contrasts (PIC) \citep{Felsenstein1985}. The test confirmed a significant
positive correlation (Pearson product moment correlation,
\(p = 0.0001225\), corrected for phylogeny using PIC) between TE content
and genome size. Additionally, genome size is correlated with TE
diversity, that is, the number of different TE superfamilies found in a
genome (Spearman, \(p \lll 0.005\)); this is also true under PIC
(Pearson, \(p = 0.033\); Fig. S1).

\begin{figure}[h!]
\begin{center}
\includegraphics[width=0.70\columnwidth]{te-coverage-vs-genome-size}
\caption[TE content is positively correlated to genome size]{{TE content
in 73arthropod genomes is positively correlated to genome assembly size
(Spearman rank correlation test, \(p \lll 0.005\)). This correlation is
also supported under phylogenetically independent contrasts
\protect\citep{Felsenstein1985} (Pearson product moment correlation, \(p
= 0.0001225\)). Dots: Individual measurements; blue line: linear
regression; grey area: confidence interval.%
}}
\end{center}
\end{figure}

\subsection{Relative contribution of different TE types to arthropod
genome
sequences}\label{relative-contribution-of-different-te-types-to-arthropod-genome-sequences}

We assessed the relative contribution of the major TE groups (LTR, LINE,
SINE retrotransposons, and DNA transposons) to the arthropod genome
composition (Fig. 1). In most species, ``unclassified'' elements, which
need further characterization, represent the largest fraction. They
contribute up to 93 \% of the total TE coverage in the mayfly
\emph{Ephemera danica} or the copepod \emph{Eurytemora affinis}.
However, in most investigated \emph{Drosophila} species the
unclassifiable elements comprise less than 25 \% and in \emph{D.
simulans} only 11 \% of the entire TE content. Disregarding these
unclassified TE sequences, LTR retrotransposons dominate the TE content
in representatives of Diptera, in some cases contributing around 50 \%
(e.g., in \emph{D. simulans}). In Hymenoptera, on the other hand, DNA
transposons are more prevalent, such as in Jerdon's jumping ant
\emph{Harpegnathos saltator}. LINE retrotransposons are most prominent
in Hemiptera and Psocodea, with the exception of the human body louse
\emph{Pediculus humanus}, whose DNA transposons contribute the major
part of the known TE content. SINE retrotransposons were found in all
insect orders, but they never constituted the majority of the genomic TE
content in any taxon in our sampling. In some lineages, such as
Hymenoptera and most dipterans, SINEs contribute only a miniscule
amount, whereas in Hemiptera and Lepidoptera the SINE coverage varies
considerably.



\subsection{Distribution of TE superfamilies in
arthropods}\label{distribution-of-te-superfamilies-in-arthropods}

We identified almost all known TE families in at least one insect
species, and many were found to be widespread and present in all
investigated species (Fig. 3, note that in this figure, TE families were
summarized in superfamilies). Especially diverse and ubiquitous are DNA
transposons, which represent 63 out of 70 identified TE superfamilies.
Among the most widespread (present in all investigated species) DNA
transposons are Academ, Chapaev and the other families in the CMC
complex, Crypton, Dada, Ginger, hAT (Blackjack, Charlie, \emph{etc.}),
Kolobok, Maverick, Harbinger, PiggyBac, Helitron (RC), Sola, TcMar
(Mariner, Tigger, \emph{etc.}), and the P element. LINE non-LTR
retrotransposons are similarly ubiquitous, though not as diverse. Among
the most widespread LINEs are CR1, Jockey, L1, L2, LOA, Penelope, R1,
R2, and RTE. Of the LTR retrotransposons, the most widespread are Copia,
DIRS, Gypsy, Ngaro, and Pao as well as endogenous retrovirus particles
(ERV). SINE elements are diverse, but show a more patchy distribution,
with only the tRNA-derived superfamily present in all investigated
species. We found the ID element in almost all species except the Asian
long horned beetle, \emph{Anoplophora glabripennis}, and the B4 element
absent from eight species. All other SINE superfamilies are absent in at
least 13 species. The Alu element was found in 48 arthropod genomes, for
example in the silkworm \emph{Bombyx mori} (Fig. 4, all Alu alignments
are shown in supplemental file 1).

\begin{figure}[h!]
\begin{center}
\includegraphics[width=\textwidth]{presence-absence-reduced-with-tree}
\caption[TE diversity in arthropod genomes]{{TE diversity in arthropod genomes: Many known TE superfamilies were
identified in almost all insect species. Presence of TE superfamilies is
shown as filled cells with the color gradient showing the TE copy number
(log11). Empty cells represent absence of TE superfamilies. The numbers
after each species name show the number of different TE superfamilies;
numbers in parentheses below clade names denote the average number of TE
superfamilies in the corresponding taxon.%
}}
\end{center}
\end{figure}

\begin{figure}[h!]
\begin{center}
\includegraphics[width=\textwidth]{Alu-in-Zootermopsis-nevadensis}
\caption[The Alu element found in \emph{Bombyx mori}]{{The Alu element
found in \emph{Bombyx mori}: Alignment of the canonical Alu sequence
from Repbase with HMM hits in the~\emph{B. mori} genome assembly. Grey
areas in the sequences are identical to the canonical Alu sequence. The
sequence names follow the pattern ``identifier:start-end(strand)'' Image
created using Geneious version 7.1 created by Biomatters. Available
from~\textbackslash{}url\{\url{https://www.geneious.com}\}
{\label{169157}}%
}}
\end{center}
\end{figure}

On average, the analyzed species harbor a mean of 54.8 different TE
superfamilies, with the locust \emph{L. migratoria} exhibiting the
greatest diversity (61 different TE superfamilies), followed by the tick
\emph{Ixodes scapularis} (60), the velvet worm \emph{Euperipatoides
rowelli} (59), and the dragonfly \emph{Ladona fulva} (59). Overall,
Chelicerata have the highest average TE superfamily diversity (56.7).
The greatest diversity among the multi-representative hexapod orders was
found in Hemiptera (55.7). The mega-diverse insect orders Diptera,
Hymenoptera, and Coleoptera display a relatively low diversity of TE
superfamilies (48.5, 51.8, and 51.8, respectively). The lowest diversity
was found in \emph{A. aegypti}, with only 41 TE superfamilies.

\subsection{Lineage-specific TE presence and absence among
insects}\label{lineage-specific-te-presence-and-absence-among-insects}

We found lineage-specific TE diversity within most insect orders. For
example, the LINE Proto1 is absent in all Hymenoptera studied, whereas
Proto2 was found in all Hymenoptera except in the ant \emph{H.
saltator}. Similarly, the DNA element Harbinger was found in all
Lepidoptera except for the silkworm \emph{B. mori}. Also within
Palaeoptera (\emph{i.e.}, mayflies, damselflies, and dragonflies), the
DNA element Harbinger is absent in \emph{E. danica}, but present in all
other species. These clade-specific absences of a TE family may be the
result of lineage-specific TE extinction events during the evolution of
the different insect orders.

We also found TE families represented only in single species of a clade.
For example, the DNA element Zisupton was found only in the wasp
\emph{Copidosoma floridanum}, but not in other Hymenoptera, and the DNA
element Novosib was found only in \emph{B. mori}, but not in other
Lepidoptera. Within Coleoptera, only the Colorado potato beetle,
\emph{Leptinotarsa decemlineata} harbors the LINE Odin. Likewise, we
found the LINE Odin among Lepidoptera only in the noctuid
\emph{Helicoverpa punctigera}. These examples of clade or lineage
specific occurrence of TEs, which are absent from other species of the
same order, could be the result of a horizontal transfer from food
species or a bacterial/viral infection.



\subsection{Lineage-specific TE activity during arthropod
evolution}\label{lineage-specific-te-activity-during-arthropod-evolution}

\begin{figure}[h!]
\begin{center}
\includegraphics[width=\textwidth]{tree-with-landscapes-and-larger-plots}
\caption[Arthropod repeat landscapes]{{Cladogram with repeat landscape
plots. The larger plots are selected representatives. The further to the
left a peak in the distribution is, the younger the corresponding TE
fraction generally is (low TE intra-family sequence divergence). In most
orders, the TE divergence distribution is similar, such as in Diptera or
Hymenoptera. The large fraction of unclassified elements was omitted for
these plots. Pal., Palaeoptera%
}}
\end{center}
\end{figure}

We further analyzed sequence divergence measured by Kimura distance
within each species-specific TE content (Fig. 5; note that for these
plots, we omitted the large fraction of unclassified elements). Within
Diptera, the most striking feature is that almost all investigated
drosophilids show a large spike of LTR retroelement proliferation
between Kimura distance 0 and around 0.08. This spike is only absent in
\emph{D. miranda}, but bi-modal in \emph{D. pseudoobscura}, with a
second peak around Kimura distance 0.15. This second peak, however, does
not coincide with the age of inversion breakpoints on the third
chromosome of \emph{D. pseudoobscura}, which are only a million years
old and have been associated with TE activity \citep{Wallace2011}. A
bi-modal distribution was not observed in any other fly species. On the
contrary, all mosquito species exhibit a large proportion of DNA
transposons which show a divergence between Kimura distance 0.02 and
around 0.3. This divergence is also present in the calyptrate flies
\emph{Musca domestica}, \emph{Ceratitis capitata}, and \emph{Lucilia
cuprina}, but absent in all acalyptrate flies, including representatives
of the \emph{Drosophila} family. Likely, the LTR proliferation in
drosophilids as well as the DNA transposon expansion in mosquitos and
other flies was the result of a lineage-specific invasion and subsequent
propagation into the different dipteran genomes.



In the calyptrate flies, Helitron elements are highly abundant,
representing 28 \% of the genome in the house fly \emph{M. domestica}
and 7 \% in the blow fly \emph{Lucilia cuprina}. These rolling circle
elements are not as abundant in acalyptrate flies, except for the
drosophilids \emph{D. mojavensis}, \emph{D. virilis}, \emph{D. miranda},
and \emph{D. pseudoobscura} (again with a bi-modal distribution). In the
barley midge, \emph{Mayetiola destructor}, DNA transposons occur across
almost all Kimura distances between 0.02 and 0.45. The same holds true
for LTR retrotransposons, although these show an increased expansion in
the older age categories at Kimura distances between 0.37 and 0.44.
LINEs and SINEs as well as Helitron elements show little occurrence in
Diptera. In \emph{B. antarctica}, LINE elements are the most prominent
and exhibit a distribution across all Kimura distances up to 0.4. This
may be a result of the overall low TE concentration in the small
\emph{B. antarctica} genome (less than 1 \%) that introduces stochastic
noise.

In Lepidoptera, we found a relatively recent SINE expansion event around
Kimura distance 0.03 to 0.05. In fact, Lepidoptera and Trichoptera are
the only holometabolous insect orders with a substantial SINE portion of
up to 9 \% in the silk worm \emph{B. mori} (mean: 3.8 \%). We observed
that in the postman butterfly, \emph{Heliconius melpomene}, the SINE
fraction also appears with a divergence between Kimura distances 0.1 to
around 0.31. Additionally, we found high LINE content in the monarch
butterfly \emph{Danaus plexippus} with a divergence ranging from Kimura
distances 0 to 0.47 and a substantial fraction around Kimura distance
0.09.

In all Coleoptera species, we found substantial LINE and DNA content
with a divergence around Kimura distance 0.1. In the beetle species
\emph{Onthophagus taurus}, \emph{Agrilus planipennis}, and \emph{L.
decemlineata}, this fraction consists mostly of LINE copies, while in
\emph{T. castaneum} and \emph{A. glabripennis} DNA elements make up the
major fraction. In all Coleoptera species, the amount of SINEs and
Helitrons is small (cf.~Fig. 1). Interestingly, \emph{Mengenilla
moldrzyki}, a representative of Strepsiptera, which was previously
determined to be the sister group of Coleoptera \citep{Niehuis2012},
shows more similarity in TE divergence distribution to Hymenoptera than
to Coleoptera, with a large fraction of DNA elements covering Kimura
distances 0.05 to around 0.3 and relatively small contributions from
LINEs.

In apocritan Hymenoptera (\emph{i.e.}, those with a wasp waist), the DNA
element divergence distribution exhibits a peak around Kimura distance
0.01 to 0.05. In fact, the TE divergence distribution looks very similar
among the ants and differs mostly in absolute coverage, except in
\emph{Camponotus floridanus}, which shows no such distinct peak.
Instead, in \emph{C. floridanus}, we found DNA elements and LTR elements
with a relatively homogeneous coverage distribution between Kimura
distances 0.03 and 0.4. \emph{C. floridanus} is also the only
hymenopteran species with a noticeable SINE proportion; this fraction's
peak divergence is around Kimura distance 0.05. The relatively TE-poor
genome of the honey bee, \emph{Apis mellifera} contains a large fraction
of Helitron elements with a Kimura distance between 0.1 and 0.35, as
does \emph{Nasonia vitripennis} with peak coverage around Kimura
distance 0.15. These species-specific Helitron appearances are likely
the result of an infection from a parasite or virus, as has been
demonstrated in Lepidoptera \citep{Coates2015}. In the (non-apocritan)
parasitic wood wasp, \emph{O. abietinus}, the divergence distribution is
similar to that in ants, with a dominant DNA transposon coverage around
Kimura distance 0.05. The turnip sawfly, \emph{A. rosae} has a large,
zero-divergence fraction of DNA elements, LINEs and LTR retrotransposons
followed by a bi-modal divergence distribution of DNA elements.

When examining Hemiptera, Thysanoptera, and Psocodea, the DNA element
fraction with high divergence (peak Kimura distance 0.25) sets the
psocodean \emph{P. humanus} apart from Hemiptera and Thysanoptera.
Additionally, the human body louse, \emph{P. humanus} exhibits a large
peak of LTR element coverage with a low divergence (Kimura distance 0).
In Hemiptera, we found DNA elements with a high coverage around Kimura
distance 0.05 instead of around 0.3, like in \emph{P. humanus}, or only
in miniscule amounts, such as in \emph{Halyomorpha halys}.
Interestingly, the three bug species \emph{H. halys}, \emph{Oncopeltus
fasciatus}, and \emph{Cimex lectularius} show a strikingly similar TE
divergence distribution which differs from that in other species of
Hemiptera. In these species, the TE landscape is characterized by a
wide-ranging distribution of LINE divergence with peak coverage around
Kimura distance 0.07. Further, they exhibit a shallow, but consistent
proportion of SINE coverage with a divergence distribution between
Kimura distance 0 and around 0.3. The other species of Hemiptera show no
clear pattern of similarity. In the water strider \emph{Gerris buenoi}
and the flower thrips \emph{Frankliniella occidentalis} as well as the
cicadellid \emph{Homalodisca vitripennis}, the Helitron elements show a
distinct coverage between Kimura distances 0 and 0.3, with peak coverage
at around 0.05 to 0.1 (\emph{F. occidentalis}, \emph{G. buenoi}) resp.
0.2 (\emph{H. vitripennis}). In both \emph{F. occidentalis} and \emph{G.
buenoi}, the divergence distribution is slightly bi-modal. In \emph{H.
vitripennis}, LINEs and DNA elements exhibit a divergence distribution
with high coverage at Kimura distances 0.02 to around 0.45. SINEs and
LTR element coverage is only slightly visible. This is in stark contrast
to the findings in the pea aphid \emph{Acyrthosiphon pisum}, where SINEs
make up the majority of the TE content and exhibit a broad spectrum of
Kimura distances from 0 to 0.3, with peak coverage at around Kimura
distance 0.05. Additionally, we found DNA elements in a similar
distribution, but showing no clear peak. Instead, LINEs and LTR elements
are distinctly absent from the \emph{A. pisum} genome, possibly as a
result of a lineage-specific extinction event.

The TE landscape in Polyneoptera is dominated by LINEs, which in the
cockroach \emph{Blattella germanica} have a peak coverage at around
Kimura distance 0.04. In the termite \emph{Zootermopsis nevadensis}, the
peak LINE coverage is between Kimura distances 0.2 and 0.4. In the
locust \emph{L. migratoria}, LINE coverage shows a broad divergence
distribution. Low-divergence LINEs show peak coverage at around Kimura
distance 0.05. All three Polyneoptera species have a small, but
consistent fraction of low-divergence SINE coverage with peak coverage
between Kimura distances 0 to 0.05 as well as a broad, but shallow
distribution of DNA element divergence.



LINEs also dominate the TE landscape in Paleoptera. The mayfly \emph{E.
danica} additionally exhibits a population of LTR elements with medium
divergence in the genome. In the dragonfly \emph{L. fulva}, we found DNA
elements of similar coverage and divergence as the LTR elements. Both TE
types have almost no low-divergence elements in \emph{L. fulva}. In the
early divergent apterygote hexapod orders Diplura (represented by the
species \emph{Catajapyx aquilonaris}) and Archaeognatha (\emph{Machilis
hrabei}), DNA elements are abundant with a broad divergence spectrum and
low-divergence peak coverage. Additionally, we found other TE types with
high coverage in low divergence regions in the genome of \emph{C.
aquilonaris} as well as SINE peak coverage at slightly higher divergence
in \emph{M. hrabei}.

The non-insect outgroup species also exhibit a highly heterogeneous TE
copy divergence spectrum. In all species, we found high coverage of
varying TE types with low divergence. All chelicerate genomes contain
mostly DNA transposons, with LINEs and SINEs contributing a fraction in
the spider \emph{Parasteatoda tepidariorum} and the tick \emph{I.
scapularis}. The only available myriapod genome, that of the centipede
\emph{Strigamia maritima}, is dominated by LTR elements with high
coverage in a low-divergence spectrum, but also LTR elements that
exhibit a higher Kimura distance. We found the same in the crustacean
\emph{Daphnia pulex}, but the TE divergence distribution in the other
crustacean species was different and consisted of more DNA transposons
in the copepod \emph{E. affinis}, or LINEs in the amphipod
\emph{Hyalella azteca}.

\section{Discussion}\label{discussion}

We used species-specific TE libraries to assess the genomic
retrotransposable and transposable element content in sequenced and
assembled genomes of arthropod species, including most extant insect
orders.

\subsection{TE content contributes to genome size in
arthropods}\label{te-content-contributes-to-genome-size-in-arthropods}

TEs and other types of DNA repeats are an omnipresent part of metazoan,
plant, as well as fungal genomes and are found in variable proportions
in sequenced genomes of different species. In vertebrates and plants,
studies have shown that TE content is a predictor for genome size
\citep{Chalopin2015,Staton2015}. For insects, this has also been
reported in clade-specific studies such as those on mosquitoes
\citep{Neafsey2014} and \emph{Drosophila} fruit flies
\citep{Sessegolo2016}. These observations lend further support to the
hypothesis that genome size is also correlated with TE content in
insects on a pan-ordinal scale.

Our analysis shows that TE content is highly variable among the
investigated insect genomes, even in comparative contexts with low
variation in genome size. Still, we found that TE content significantly
contributes to genome size in insects. The strength of this contribution
is significantly different between hemimetabolous and holometabolous
insects. These results are in line with prior studies on insects with a
more limited taxon sampling reporting a clade-specific correlation
between TE content and genome size
\citep{Vieira1999,Vieira2002,Kidwell2000,Honeybee2006,Bosco2007,Sessegolo2016}.
These findings further support the hypothesis that TEs are a major
factor in the dynamics of genome size evolution in Eukaryotes. While
differential TE activity apparently contributes to genome size variation
\citep{Petrov2001,Kidwell2002,Agren2011}, whole genome duplications
(suggested by integer-sized genome size variations in some
representatives of Hymenoptera, but unconfirmed so far), segmental
duplications, deletions, and other repeat proliferation
\citep{Parfrey2008} could contribute as well. This variety of
influencing factors potentially explains the range of dispersion in the
correlation.

The high range of dispersion in the correlation of TE content and genome
size is most likely also amplified by heterogeneous underestimates of
the genomic TE coverage. Most of the genomes were sequenced and
assembled using different methods, and with insufficient sequencing
depth and/or older assembly methods; the data are therefore almost
certainly incomplete with respect to repeat-rich regions. Assembly
errors and artifacts also add a possible error margin, as assemblers
still struggle to reconstruct repeat regions accurately from short reads
\citep{Schatz2010}. Additionally, RepeatMasker is known to underestimate
the genomic repeat content \citep{deKoning2011}. By combining
RepeatModeler to infer the species-specific repeat libraries and
RepeatMasker to annotate the species-specific repeat libraries in the
genome assemblies, our methods are purposefully conservative and may
have missed some TE types, or ancient and highly divergent copies.

This underestimation of the TE content notwithstanding, we found many TE
families that were previously thought to be restricted to, for example,
primates, such as the SINE family Alu \citep{Kriegs2007} and the LINE
family L1 \citep{Liu2003}, or to fungi, such as Tad1
\citep{Cambareri1994}. Essentially, most known superfamilies were found
in the investigated insect genomes (\emph{cf}. Fig. 3) and additionally,
we identified highly abundant unclassifiable TEs in all insect species.
These observations suggest that the insect mobilome (the entirety of
mobile DNA elements) is more diverse than the well characterized
vertebrate mobilome \citep{Chalopin2015} and requires more exhaustive
characterization. We were able to reach these conclusions by relying on
two essential non-standard analyses. First, our annotation strategy of
\emph{de novo} repeat library construction and classification according
to the RepBase database was more specific to each genome than the
default RepeatMasker analysis using only the RepBase reference library.
The latter approach is usually done when releasing a new genome assembly
to the public. The second difference between our approach and the
conventional application of the RepBase library was that we used the
entire Metazoa-specific section of RepBase instead of restricting our
search to Insecta. This broader scope allowed us to annotate TEs that
were previously unknown from insects, and that would otherwise have been
overlooked. Additionally, by removing results that matched non-TE
sequences in the NCBI database, our annotation becomes more robust
against false positives. The enormous previously overlooked diversity of
TEs in insects does not seem to be surprising given the geological age
and species richness of this clade. Insects originated more than 450
million years ago \citep{Misof2014} and represent over 80 \% of the
described metazoan species \citep{Grimaldi2005}. It remains to be
analyzed in detail, however, whether insect TE diversity evolved
independently within insects or is the result of multiple TE
introgression into insect genomes. Further investigations will also show
whether there is a connection between TE diversity or abundance and
clade-specific genetic and genomic traits, such as the sex determination
system (\emph{e.g.}, butterflies have Z and W chromosomes instead of X
and Y \citep{Traut1997}) or the composition of telomeres, which have
been shown in \emph{D. melanogaster} to exhibit a high density of TEs
\citep{Levis1993}, whereas telomeres in other insects consist mostly of
simple repeats.

Our results show that virtually all known TE classes are present in all
investigated insect genomes.  However, a large part of the TEs we
identified remains unclassifiable despite the diversity of metazoan TEs
in the reference library RepBase.  This abundance of unclassifiable TEs
suggests that the insect TE repertoire requires more exhaustive
characterization and that our understanding of the insect mobilome is
far from complete.

It has been hypothesized that population-level processes might
contribute to TE content differences and genome size variation in
vertebrates \citep{Lynch2003}. In insects, it has been shown that TE
activity also varies on the population level, for example in the genomes
of \emph{Drosophila} spp. \citep{Perrat2013,Li2013,Blumenstiel2013} or
in the genome of the British peppered moth \emph{Biston betularia}, in
which a tandemly repeated TE confers an adaptive advantage in response
to short-term environmental changes \citep{Hof2016}. The TE activity
within populations is expected to leave footprints in the nucleotide
sequence diversity of TEs in the genome as recent bursts of TEs should
be detectable by a large number of TE sequences with low sequence
divergence. \citet{Struchiner2009} proposed a coalescent model for TE
amplification dynamics based on \emph{Aedes} and \emph{Anopheles}
mosquito genomes.



To explain TE proliferation dynamics, two different models of TE
activity have been proposed: the equilibrium model and the burst model.
In the equilibrium model, TE proliferation and elimination rates are
more or less constant and cancel each other out at a level that is
different for each genome \citep{Charlesworth1983}. In this model,
differential TE elimination rate contributes to genome size variation
when TE activity is constant. This model predicts that in species with a
slow rate of DNA loss, genome size tends to increase
\citep{Petrov2010,Sun2011}.  In the burst model, TEs do not proliferate
at a constant rate, but rather in high copy rate bursts following a
period of inactivity \citep{Blumenstiel2013}. These bursts can be TE
family specific. Our analysis of TE landscape diversity (see below),
supports the burst hypothesis. In almost every species we analyzed,
there is a high proportion of abundant TE sequences with low sequence
divergence and the most abundant TEs are different even among closely
related species. It was hypothesized that TE bursts and counteracting
lineage-specific host defense mechanisms such as TE silencing
\citep{LeRouzic2006} have resulted in differential TE contribution to
genome size.

\subsection{TE landscape diversity in
insects}\label{te-landscape-diversity-in-insects}

In vertebrates, it is possible to trace lineage-specific contributions
of different TE types \citep{Chalopin2015}. In insects, however, the TE
composition is significantly, but weakly correlated to genome size.
Instead, we can show that major differences both in TE abundance and
diversity exist between species of the same lineage (Fig. 3). Using the
Kimura nucleotide sequence distance, we observe distinct variation, but
also similarities, in TE composition and activity between insect orders
and among species of the same order. The number of recently active
elements can be highly variable, such as LTR retrotransposons in fruit
flies or DNA transposons in ants (Fig. 5). On the other hand, the shape
of the TE coverage distributions can be fairly similar among species of
the same order; this is particularly visible in Hymenoptera and Diptera.
These findings suggest lineage-specific similarities in TE elimination
mechanisms; possibly shared efficacies in the piRNA pathway that
silences TEs during transcription \citep{LeThomas2013} that evolved once
in these lineages. Another possible explanation would be recent
horizontal transfers from, for example, parasite to host species (see
below).

\subsection{Can we infer an ancestral insect mobilome in the face of
massive horizontal TE
transfer?}\label{can-we-infer-an-ancestral-insect-mobilome-in-the-face-of-massive-horizontal-te-transfer}

In a purely vertical mode of TE transmission, the genome of the last
common ancestor (LCA) of insects can be assumed to possess a superset of
the TE superfamilies present in extant insect species. As many TE
families appear to have been lost due to lineage-specific TE extinction
events, the ancestral TE repertoire may have been even more extensive
compared with the TE repertoire of extant species and might have
included almost all known metazoan TE superfamilies such as the CMC
complex, Ginger, Helitron, Mavericks, Jockey, L1, Penelope, R1, DIRS,
Ngaro, and Pao. Many SINEs found in extant insects were most likely part
of the ancestral mobilome as well, for example Alu, which was previously
thought to be restricted to primates \citep{Deininger_2011}, and MIR.

In contrast to a vertical mode of transmission, horizontal gene
transfers, common phenomenona among prokaryotes and widely occurring in
plants, are rather rare in vertebrates \citep{Syvanen2012,Wallau2012},
but have been described in Lepidoptera \citep{Sormacheva2012} and other
insects \citep{Nakabachi2015}. Recently, a study uncovered large-scale
horizontal transfer of TEs (horizontal transposon transfer, HTT) among
insects \citep{Peccoud2017} and makes this mechanism even more likely to
be the source of inter-lineage similarities in insect genomic TE
composition.  In the presence of massive HTT, an ancestral mobilome
might be impossible to infer because the effects of HTT overshadow the
result of vertical TE transfer. It remains to be analyzed in detail
whether the high diversity of the insect mobilomes can be better
explained by massive HTT events.

\section{Conclusions}\label{conclusions}

The present study provides an overview of the diversity and evolution of
TEs in the genomes of major lineages of extant insects.  The results
show that there is large intra- and inter-lineage variation in both TE
content and composition.  This, and the highly variable age distribution
of individual TE superfamilies, indicate a lineage-specific burst-like
mode of TE proliferation in insect genomes.  In addition to the complex
composition patterns that can differ even among species of the same
genus, there is a large fraction of TEs that remain unclassified, but
often make up the major part of the genomic TE content, indicating that
the insect mobilome is far from completely characterized.  This study
provides a solid baseline for future comparative genomics research.  The
functional implications of lineage-specific TE activity for the
evolution of genome architecture remain the focus of future
investigations.


\subsection{Availability of data and
material}\label{availability-of-data-and-material}

All genome assembly sources are listed in supplemental table S1.

\subsection{Funding}\label{funding}

BM, MP, and ON were supported by the Leibniz Graduate School on Genomic
Biodiversity Research and by the German Research Foundation (DFG, MI
649/16--1; NI1387/3-1). RAG and SR were supported by the National
Institutes of Health (U54 HG003273 awarded to RAG). AK and DA were
supported by the European Research Council (ERC-CoG \#616346 to AK).

\subsection{Authors' contributions}\label{authors-contributions}

BM, MP, and ON conceived the study. MP performed all analyses. BM and MP
interpreted the results and wrote the manuscript draft. AK, DA, GM, LH,
and ON collected specimens and performed laboratory procedures including
RNA/DNA extraction. RAG and SR co-ordinated, sequenced, assembled and
made available genome reference sequences of species within the i5K
pilot. All authors read, contributed to, and approved the final
manuscript.

\subsection{Acknowledgements}\label{acknowledgements}

We thank the i5k pilot consortium and the staff of the Baylor College of
Medicine Human Genome Sequencing Center (BCM-HGSC) for the generation of
and access to pre-publication data. We further thank Severine Viala for
her help with inbreeding of and DNA extraction from \emph{G. buenoi}. We
are grateful to Dorith Rotenberg for coordinating the \emph{F.
occidentalis} genome project as part of the BCM-HGSC i5k pilot
initiative, and for providing pre-publication data. We gratefully
acknowledge the coordinators of the i5k \emph{Blattella} genome project
for providing gDNA samples and enabling the genome assembly. We thank
the Agricultural Research Service of the United States Department of
Agriculture (USDA-ARS) for making the unpublished genome assembly of
\emph{H. halys} available for analysis. Finally, we thank John Oakeshott
and Karl Gordon at the Commonwealth Scientific and Industrial Research
Organisation (CSIRO) for pre-publication access to the \emph{H.
punctigera} genome.

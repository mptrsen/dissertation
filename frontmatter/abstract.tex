%!TEX root = ../dissertation.tex
% the abstract

\section*{Chapter \ref{cha:mobilome}}
Transposable elements (TEs) are a major component of metazoan genomes
and are associated with a variety of mechanisms that shape genome
architecture and evolution. Despite the ever-growing number of insect
genomes sequenced to date, our understanding of the diversity and
evolution of insect TEs remains poor. Here, we present a standardized
characterization and an order-level comparison of insect TE repertoires,
encompassing 72 species. The insect TE repertoire contains TEs of almost
every class previously described, and in some cases even TEs previously
reported only from vertebrates and plants. Additionally, we identified a
large fraction of unclassifiable TEs. We found high variation in TE
content, ranging from less than 6 \% in the antarctic midge (Diptera),
the honey bee and the turnip sawfly (Hymenoptera) to more than 58 \% in
the malaria mosquito (Diptera) and the migratory locust (Orthoptera),
and a possible relationship between the content and diversity of TEs and
the genome size. While most insect orders exhibit a characteristic TE
composition, we also observed intraordinal differences, e.g., in
Diptera, Hymenoptera, and Hemiptera. Our findings shed light on common
patterns and reveal lineage-specific differences in content and
evolution of TEs in insects. We anticipate our study to provide the
basis for future comparative research on the insect TE repertoire.%


\section*{Chapter \ref{cha:orthograph}}
Orthology characterizes genes of different organisms that arose from a
single ancestral gene via speciation, in contrast to paralogy, which is
assigned to genes that arose via gene duplication. An accurate orthology
assignment is a crucial step for comparative genomic studies.
Orthologous genes in two organisms can be identified by applying a
so-called reciprocal search strategy, given that complete information of
the organisms' gene repertoire is available. In many investigations,
however, only a fraction of the gene content of the organisms under
study is examined (e.g., RNA sequencing).  Here, identification of
orthologous nucleotide or amino acid sequences can be achieved using a
graph-based approach that maps nucleotide sequences to genes of known
orthology. Existing implementations of this approach, however, suffer
from algorithmic issues that may cause problems in downstream analyses.

We present a new software pipeline, Orthograph, that addresses and
solves the above problems and implements useful features for a wide
range of comparative genomic and transcriptomic analyses.  Orthograph
applies a best reciprocal hit search strategy using profile hidden
Markov models and maps nucleotide sequences to the globally best
matching cluster of orthologous genes, thus enabling researchers to
conveniently and reliably delineate orthologs and paralogs from
transcriptomic and genomic sequence data. We demonstrate the performance
of our approach on \emph{de novo}-sequenced and assembled transcript
libraries of 24 species of apoid wasps (Hymenoptera: Aculeata) as well
as on published genomic datasets.

With Orthograph, we implemented a best reciprocal hit approach to
reference-based orthology prediction for coding nucleotide sequences
such as RNAseq data. Orthograph is flexible, easy to use, open source
and freely available at \url{https://mptrsen.github.io/Orthograph}.
Additionally, we release 24 \emph{de novo}-sequenced and assembled
transcript libraries of apoid wasp species.

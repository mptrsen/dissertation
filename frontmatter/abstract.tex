%!TEX root = ../dissertation.tex

% the abstract

This thesis presents comparative genomics studies in insects as well as
bioinformatics software development. Its empirical research part is
focused mainly on mobile genetic elements, also termed transposable
elements. The data basis contains datasets from public repositories, a
rich and often underexplored source of information on genomic
biodiversity. Transposable elements in particular are often neglected
when the results of a genome sequencing study are published, although
they make up a major part of virtually every eukaryotic genome. 

After a general introduction in Chapter \ref{cha:general-introduction}, I
characterize and compare the transposable element repertoire of 73
arthropod species in Chapter \ref{cha:mobilome} and find that it
correlates to genome size in both abundance and diversity. In Chapter
\ref{cha:dynamics}, I study the effect of transposable elements on the
evolution of genome size in more detail and on an expanded dataset of 96
species. Finally, in Chapter \ref{cha:orthograph} I present a software
pipeline for delineating orthology among coding nucleotide sequences, an
essential tool for many comparative and phylogenetic studies.

\section*{Chapter \ref{cha:mobilome}}

Transposable elements (TEs) are a major component of metazoan genomes
and are associated with a variety of mechanisms that shape genome
architecture and evolution. Despite the ever-growing number of insect
genomes sequenced to date, our understanding of the diversity and
evolution of insect TEs remains poor. Here, we present a standardized
characterization and an order-level comparison of arthropod TE
repertoires, encompassing 62 insect and 11 outgroup species. The insect
TE repertoire contains TEs of almost every class previously described,
and in some cases even TEs previously reported only from vertebrates and
plants. Additionally, we identified a large fraction of unclassifiable
TEs. We found high variation in TE content, ranging from less than 6 \%
in the antarctic midge (Diptera), the honey bee and the turnip sawfly
(Hymenoptera) to more than 58 \% in the malaria mosquito (Diptera) and
the migratory locust (Orthoptera), and a possible relationship between
the content and diversity of TEs and the genome size. While most insect
orders exhibit a characteristic TE composition, we also observed
intraordinal differences, e.g., in Diptera, Hymenoptera, and Hemiptera.
Our findings shed light on common patterns and reveal lineage-specific
differences in content and evolution of TEs in insects. We anticipate
our study to provide the basis for future comparative research on the
insect TE repertoire.%

\section*{Chapter \ref{cha:dynamics}}

Genome size in insects displays inter-specific variation in excess of
130-fold, a range only paralleled in the metazoan phylum by amphibians.
In general, these inter-specific differences seem to be best explained
by differential rates of transposable element (TE) accumulation.  In
fact, we observe that TE accumulation rates are lineage-specific and
that major insect clades have distinct TE age distributions.  Given this
observation, we hypothesize that evolutionarily younger insect lineages
should have more TEs that are older than the insect lineage itself.  To
test this hypothesis, we infer ancient and lineage-specific TE
insertions, and quantify genome size increase and decrease in 96
arthropod species from 18 major insect orders, spanning a geological age
range of around 400 million years.  Our analysis reveals that most
insect lineages appear to have a specific rate of TE accumulation that
is correlated with genome size, along with a distinct, clade-specific
and TE class dependent TE age distribution.  Additionally,
lineage-specific rates of genome size reduction appear to counteract
genome expansion through TE activity.  Our results are inconsistent with
a general "accordion" model of genome size dynamics in eukaryotes,
therefore we suggest that TE management in insects is fundamentally
different than in vertebrates.  We propose that in the face of
burst-like TE proliferation events, clade-specific rates of genome size
reduction strongly influence the large variation in extant insect genome
sizes.

\section*{Chapter \ref{cha:orthograph}}

Orthology characterizes genes of different organisms that arose from a
single ancestral gene via speciation, in contrast to paralogy, which is
assigned to genes that arose via gene duplication. An accurate orthology
assignment is a crucial step for comparative genomic studies.
Orthologous genes in two organisms can be identified by applying a
so-called reciprocal search strategy, given that complete information of
the organisms' gene repertoire is available. In many investigations,
however, only a fraction of the gene content of the organisms under
study is examined (e.g., RNA sequencing).  Here, identification of
orthologous nucleotide or amino acid sequences can be achieved using a
graph-based approach that maps nucleotide sequences to genes of known
orthology. Existing implementations of this approach, however, suffer
from algorithmic issues that may cause problems in downstream analyses.

We present a new software pipeline, Orthograph, that addresses and
solves the above problems and implements useful features for a wide
range of comparative genomic and transcriptomic analyses.  Orthograph
applies a best reciprocal hit search strategy using profile hidden
Markov models and maps nucleotide sequences to the globally best
matching cluster of orthologous genes, thus enabling researchers to
conveniently and reliably delineate orthologs and paralogs from
transcriptomic and genomic sequence data. We demonstrate the performance
of our approach on \emph{de novo}-sequenced and assembled transcript
libraries of 24 species of apoid wasps (Hymenoptera: Aculeata) as well
as on published genomic datasets.

With Orthograph, we implemented a best reciprocal hit approach to
reference-based orthology prediction for coding nucleotide sequences
such as RNAseq data. Orthograph is flexible, easy to use, open source
and freely available at \url{https://mptrsen.github.io/Orthograph}.
Additionally, we release 24 \emph{de novo}-sequenced and assembled
transcript libraries of apoid wasp species.
